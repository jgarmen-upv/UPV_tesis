% ---------------------------------------------------------------------
% ---------------------------------------------------------------------
% Configuración general

% For PDF/A documents, conflict with hyperref package not solved
%\usepackage[a-1b]{pdfx}

% To use "sen" as "sin" (sinus) for spanish/valencian maths
\newcommand{\sen}{\on{sen}}
\usepackage{csquotes}
% Quotes for valencian/catalen not defined, set to same as spanish
%\DeclareQuoteAlias{spanish}{catalan}

% Some general packages 
\usepackage{array,booktabs}
\usepackage{tabularx,longtable,multicol}
\usepackage{amssymb}

% ---------------------------------------------------------------------
% ---------------------------------------------------------------------
% Custom commands and packages JGM

% To use boldmath in headers
\usepackage{bm} 

% Change length between lines in math mode
\setlength{\jot}{12pt}

% Make the acronym list
\usepackage[acronym,shortcuts]{glossaries}

% Chapter 1
\newacronym{sm}{SM}{Standard Model}
\newacronym{cr}{CR}{Cosmic Ray}
\newacronym{cmb}{CMB}{Cosmic Microwave Background}
\newacronym{pmns}{PMNS}{Pontecorvo-Maki-Nakagawa-Sakata matrix for lepton mixing}
\newacronym{gzk}{GZK}{Greisen-Zatsepin-Kuzmin}

\newacronym{snr}{SNR}{Shell-type Supernova Remnant}
\newacronym{pwne}{PWNe}{Pulsar Wind Nebulae}
\newacronym{xrb}{XRB}{X-Ray Binary}
\newacronym{grb}{GRB}{Gamma-Ray Burst}
\newacronym{agn}{AGN}{Active Galactic Nucleus}

\newacronym{cc}{CC}{Charged Current}
\newacronym{nc}{NC}{Neutral Current}

\newacronym{pmt}{PMT}{Photomultiplier Tube}
\newacronym{om}{OM}{Optical Module}
\newacronym{dom}{DOM}{Digital Optical Module}
\newacronym{du}{DU}{Detection Unit}

\newacronym{dumand}{DUMAND}{Deep Underwater Muon And Neutrino Detector}
\newacronym{amanda}{AMANDA}{Antarctic Muon And Neutrino Detector Array}
\newacronym{nestor}{NESTOR}{Neutrino Extended Submarine Telescope with Oceanographic Research}
\newacronym{nemo}{NEMO}{Neutrino Ettore Majorana Observatory}
\newacronym{p-one}{P-ONE}{Pacific Ocean Neutrino Experiment}
\newacronym{trident}{TRIDENT}{TRopIcal DEep-sea Neutrino Telescope}
\newacronym{b-gvd}{Baikal-GVD}{Baikal Gigaton Volume Detector}
\newacronym{ic}{IceCube}{IceCube experiment}
\newacronym{sk}{Super-K}{Super-Kamiokande experiment}
\newacronym{orca}{ORCA}{Oscillation Research with Cosmics in the Abyss}
\newacronym{arca}{ARCA}{Astroparticle Research with Cosmics in the Abyss}


% Chapter 2

\newacronym{sl}{SL}{Single-Line (event)}

\newacronym{ai}{AI}{Artificial Intelligence}
\newacronym{dl}{DL}{Deep Learning}
\newacronym{ml}{ML}{Machine Learning}
\newacronym{tl}{TL}{Transfer Learning}

\newacronym{cnn}{CNN}{Convolutional Neural Network}
\newacronym{dcn}{DCN}{Deep Convolutional Network}
\newacronym{ffn}{FFN}{Feed-Forward Network}
\newacronym{mdn}{MDN}{Mixture Density Network}

\newacronym{pca}{PCA}{Principal Component Analysis}
\newacronym{relu}{ReLU}{Rectified Linear Unit}
\newacronym{elu}{ELU}{Exponential Linear Unit}
\newacronym{mle}{MLE}{Maximum Likelihood Estimation}
\newacronym{mae}{MAE}{Mean Absolute Error}
\newacronym{mse}{MSE}{Mean Squared Error}

\newacronym{cpu}{CPU}{Central Processing Unit}
\newacronym{gpu}{GPU}{Graphics Processing Unit}
\newacronym{io}{I/O}{Input/Output}
\newacronym{htc}{HTC}{High-Throughput Computing}
\newacronym{slurm}{SLURM}{Simple Linux Utility for Resource Management software}
\newacronym{in2p3}{IN2P3}{Institut National de Physique Nucléaire et de Physique des Particules}


% Chapter 3

\newacronym{mond}{MOND}{MOdified Newton Dynamics}
\newacronym{wmap}{WMAP}{Wilkinson Microwave Anisotropy Probe}

\newacronym{cdm}{CDM}{Cold Dark Matter}
\newacronym{macho}{MACHO}{MAssive Compact Halo Object}
\newacronym{bbn}{BBN}{Big Bang Nucleosynthesis}
\newacronym{pbh}{PBH}{Primordial Black Holes}

\newacronym{cp}{CP}{Charge-Parity}
\newacronym{qcd}{QCD}{Quantum ChromoDynamics}
\newacronym{susy}{SUSY}{SUperSYmmetric}
\newacronym{mssm}{MSSM}{Minimal Supersymmetric Standard Model}
\newacronym{lsp}{LSP}{Lightest Supersymmetric Particle}
\newacronym{lhc}{LHC}{Large Hadron Collider}
\newacronym{cern}{CERN}{Conseil Européen pour la Recherche Nucléaire}

\newacronym{sd}{SD}{Spin-Dependent}
\newacronym{si}{SI}{Spin-Independent}

\newacronym{tpc}{TPC}{Time Projection Chamber}

\newacronym{ams}{AMS}{Alpha Magnetic Spectrometer}
\newacronym{hess}{HESS}{High Energy Stereoscopic System}
\newacronym{magic}{MAGIC}{Major Atmospheric Gamma-ray Imaging Cherenkov}
\newacronym{xenon}{XENON}{XENON experiment}
\newacronym{lux}{LUX}{Large Underground Xenon experiment}
\newacronym{zeplin}{ZEPLIN}{ZonEd Proportional scintillation in LIquid Noble gases}
\newacronym{lz}{LZ}{LUX-ZEPLIN experiment}
\newacronym{darkside}{DarkSide}{DarkSide direct-detection experiment}
\newacronym{pico}{PICO}{PICO bubble chamber experiment}
\newacronym{cresst}{CRESST}{Cryogenic Rare Event Search with Superconducting Thermometers}
\newacronym{supercdms}{SuperCDMS}{Super Cryogenic Dark Matter Search}
\newacronym{EDELWEISS}{EDELWEISS}{Expérience pour DEtecter Les WIMPs En SIte Souterrain}

\newacronym{wimpsim}{WimpSim}{WIMP Simulation software}
\newacronym{darksusy}{DarkSUSY}{Supersymmetric Dark Matter simulation software}

\makeglossaries
% Make the acronym list print as a chapter-level heading
\setglossarysection{chapter}
% Bold acronym names, keep descriptions normal
\newglossarystyle{longbold}{%
  \setglossarystyle{long}% base on the standard 'long' style
  \renewcommand*{\glsnamefont}[1]{\textbf{##1}}% make the entry name bold
}

% Para incluir la portada como PDF (y otros PDFs si fuera necesario)
\usepackage{pdfpages}

% Para usar multirow in tables
\usepackage{multirow, makecell}

% Para poner celdas de colores en las tablas
\usepackage[table]{xcolor}  % Permite usar \cellcolor en tablas
\usepackage{colortbl}       % Opcional, pero mejora compatibilidad con tablas

% Comandos para notas y ocultar bloques de texto
\newcommand{\hide}[1]{}

% Para poner notas en el texto
\usepackage[colorinlistoftodos,prependcaption]{todonotes}
\newif\ifdraft

%\drafttrue   % pon \draftfalse para ocultarlos en la versión final
\draftfalse

\ifdraft
  \newcommand{\juan}[1]{\todo[color=blue!20,inline]{JGM: #1}}
  \newcommand{\salva}[1]{\todo[color=green!20,inline]{SA: #1}}
  \newcommand{\miquel}[1]{\todo[color=orange!25,inline]{MA: #1}}
\else
  \newcommand{\juan}[1]{}
  \newcommand{\salva}[1]{}
  \newcommand{\miquel}[1]{}
\fi

% Para usar la fuente \mathds
\usepackage{dsfont}

% Para rellenar trozos sin escribir
\usepackage{lipsum}

% Para usar lineas punteadas verticales en las tablas (|c:c|)
\usepackage{arydshln}

% Para usar subfiguras
\usepackage[labelfont=bf]{subcaption}
\renewcommand\thesubfigure{\Alph{subfigure}}

% Traducción del título del abstract
\addto\captionsenglish{\renewcommand{\abstractname}{Abstract}}
\addto\captionsspanish{\renewcommand{\abstractname}{Resumen}}
\addto\captionscatalan{\renewcommand{\abstractname}{Resum}}

% Entorno abstract personalizado tipo recomendado (como un capitulo)
\newenvironment{myabstract}{
  \chapter*{\abstractname}
  \addcontentsline{toc}{chapter}{\abstractname}
}{}

% Entorno abstract personalizado tipo report
\newenvironment{abstract}{
  %\cleardoublepage
  \clearpage
  \null\vfill
  \begin{center}%
    \bfseries \abstractname
  \end{center}}%
 {\vfill\null}

% Change subsubsections to not be in italic shape and being bold
\titleformat{\subsubsection}
  [hang]
  {\vspace{2ex}\raggedright\tolerance=10000\hyphenpenalty=10000}
  {\normalsize\bfseries\thesubsubsection}
  {1em}
  {\normalsize\bfseries}
  [\vspace{-0.75ex}]

% Change subsections to not be in italic shape
\titleformat{\subsection}
    [hang]
    {\vspace{1.5ex}\raggedright\tolerance=10000\hyphenpenalty=10000}
    {\fontsize{10.5}{12.5}\bfseries\thesubsection}
    {1em}
    {\fontsize{10.5}{12.5}\bfseries}
    [\vspace{-1ex}]

% ---------------------------------------------------------------------
% ---------------------------------------------------------------------
% ---------------------------------------------------------------------
% Bibliografía (modificar al gusto)

\usepackage[
	url      = false,
    eprint   = false,
    isbn     = false,
	style    = ieee, %numeric, %apa %authoryear
    citestyle = numeric-comp,
	hyperref = true,
	backref  = true,
	backend  = biber,
    sorting  = none,
    maxnames = 2,
    minnames = 2,
    giveninits = true,
	]{biblatex}

% To avoid reading the abstract in biber, avoid erros due to Unicode
\DeclareSourcemap{
  \maps[datatype=bibtex]{
    \map{
      % Remove abstract and languages to avoid bad printing
      \step[fieldset=abstract, null]
      \step[fieldset=language, null]
    }
    \map[overwrite=true]{
      % Only for Articles
      \pertype{Article}
      % Continue only if 'collaboration' exists (final):
      \step[fieldsource=collaboration, fieldtarget=author, final]
    }
  }
}


% ---------------------------------------------------------------------
% ---------------------------------------------------------------------
% Documento electrónico

\usepackage{imakeidx}
\makeindex


% Print boolean to change link colors. Set to true for printing, so links are black
\newif\ifprint
\printfalse
%\printtrue

% Define colors for links, internal references, etc
\colorlet{colorEnlace}{black}

\ifprint
    \colorlet{colorRef}{black}
    \colorlet{colorURL}{black}
\else
    \colorlet{colorRef}{blue}
    \colorlet{colorURL}{cyan}
\fi

% To allow URL breaks in many characters (just needed if overfull \hbox happens due to URLs)
% NOT TESTED
%\usepackage{xurl}

\usepackage[
	colorlinks,
    linkcolor=colorRef,
	citecolor=colorRef,
	urlcolor=colorURL,
    bookmarksnumbered,
	breaklinks,
	]{hyperref}

% Para usar \bookmarksetup % Debe estar siempre detras de hyperref
\usepackage{bookmark} % <- justo después de hyperref



% ---------------------------------------------------------------------
% ---------------------------------------------------------------------
% ---------------------------------------------------------------------
% ---------------------------------------------------------------------
% ---------------------------------------------------------------------
% ---------------------------------------------------------------------
% Below this point, I have not used anything of this.
% It was all declared by "editorial UPV", so it should work fine
% ---------------------------------------------------------------------
% ---------------------------------------------------------------------
% ---------------------------------------------------------------------
% ---------------------------------------------------------------------
% ---------------------------------------------------------------------
% ---------------------------------------------------------------------


% ---------------------------------------------------------------------
% ---------------------------------------------------------------------
% Expresión de unidades según el Sistema Internacional, monedas

\usepackage{eurosym}
\usepackage{siunitx}

\ifenglish
	\sisetup{output-decimal-marker={.}}
\else
	\sisetup{output-decimal-marker={,}}
\fi

\DeclareSIUnit[number-unit-product = {\;}] \EURO{\geneuro}


% ---------------------------------------------------------------------
% ---------------------------------------------------------------------
% Por compatibilidad con versiones anteriores de la plantilla
	
\newcommand{\incluyeGrafico}[2][]{\includegraphics[#1]{#2}} % Por compatibilidad

% ---------------------------------------------------------------------

\newcommand{\ingles}[1]{\textit{#1}}

% ------------------------------------------------------------------------

\usepackage{xspace}

\newcommand{\angles}[1]{\textit{#1}\/}
\newcommand{\miUrl}[1]{{\small%
	%\texttt%
	{\underline{#1}}}}

\newcommand{\matlabr}{{\sc Matlab}$^\circledR$\xspace}
\newcommand{\simulinkr}{\textit{Simulink}$^\circledR$\xspace}
\newcommand{\matlab}{{\textsc{Matlab}}\xspace}
\newcommand{\simulink}{\textit{Simulink}\xspace}

\newcommand{\scr}{\textit{script\/}\xspace}
\newcommand{\scrs}{\textit{scripts\/}\xspace}

% ------------------------------------------------------------------------

\definecolor{griset}{rgb}{.925, .925, .925}


\newsavebox{\mybox}
\newenvironment{parrafoDestacado}
    {%
    \fboxsep = 2ex
    \fboxrule = .4pt
    \begin{lrbox}{\mybox}%
    \begin{minipage}{.85\textwidth-2\fboxsep}\itshape\parskip=2ex
    }
    {%
    \end{minipage}
    \end{lrbox}%
    \begin{flushright}
        \colorbox{griset}{\usebox{\mybox}}%
        %\fcolorbox{black}{griset}{\usebox{\mybox}}%
    \end{flushright}
    }


% ------------------------------------------------------------------------
% Resumen del capítulo


\newsavebox{\myboxb}
\newenvironment{Resumen}
    {%
    \vspace*{-2.0cm}
    \fboxsep = 0pt
    \fboxrule = 0pt
    \begin{lrbox}{\myboxb}%
    \begin{minipage}{.85\textwidth}\itshape\parskip=2ex\parindent=2em
    }
    {%
    \end{minipage}
    \end{lrbox}%
    \begin{flushright}
        \usebox{\myboxb}%
    \end{flushright}
    \vspace{0.5cm}
    }

% ---------------------------------------------------------------------
% ---------------------------------------------------------------------
% Símbolos matemáticos

\newcommand{\on}{\operatorname}

% ---------------------------------------------------------------------
% ---------------------------------------------------------------------
% Teoremas y ejemplos

\ifcastellano
	\newtheorem{teorema}{\upshape\bfseries Teorema}[section]
	\newtheorem{lema}{\mdseries\scshape Lema}[section]
	\newtheorem{proposicion}{\upshape\bfseries Proposición}[section]
	\newtheorem{ejemplo}{\bfseries\scshape Ejemplo}[section]
\fi

\ifvalencia % Es mantenen els mateixos noms per compatibilitat, però l'autor els pot personalitzar
	\newtheorem{teorema}{\upshape\bfseries Teorema}[section]
	\newtheorem{lema}{\mdseries\scshape Lema}[section]
	\newtheorem{proposicion}{\upshape\bfseries Proposició}[section]
	\newtheorem{ejemplo}{\bfseries\scshape Exemple}[section]
\fi

\ifenglish
	\newtheorem{teorema}{\upshape\bfseries Theorem}[section]
	\newtheorem{lema}{\mdseries\scshape Lemma}[section]
	\newtheorem{proposicion}{\upshape\bfseries Proposition}[section]
	\newtheorem{ejemplo}{\bfseries\scshape Example}[section]
\fi
